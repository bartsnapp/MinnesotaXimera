\documentclass[number]{ximera}

\title{Solving Triangles}

\begin{document}

\maketitle

In this activity, you will be given three parts of a triangle (side lengths and/or angle measures) and will be asked to place points to produce a triangle that has these three parts. The questions are intentionally open ended, there may be many different ways to produce triangles with the given parts. When working with your group members, you should try to produce original answers. For example, if your group members put their red points high, have your red point low.

\bigskip

Two triangles are the same if they are congruent. That is, two triangles are the same if you can move one triangle to line up with another by sliding, rotating and/or flipping. Another way to describe this is if one student lists the side lengths and angle measures in order around the triangle, they should match up, in the same order with a congruent version. The congruent version may start in a different location, and may go clockwise instead of counterclockwise.

In the interactive window below, points $A$, $B$ and $C$ can be moved by using the mouse. Simply click-and-drag a point to a new location. You can also move a point by moving the slider bars using the same click-and-drag technique. $xA$ controls the $x$-coordinate of $A$, and $yA$ controls the $y$-coordinate of $A$. The same is true for points $B$ and $C$. You may also move the slider bars in increments of 0.01 by slowly clicking on the slider bar. For example, clicking on the $yC$ control bar to the right of the circle indicator will increase the $y$-coordinate of point $C$, moving point $C$ up slightly. You are limited to $x$ and $y$-coordinates between -5 and 5. You will use the same interactive window for all nine parts of this exploration.

\HCode{<iframe scrolling="no" src="https://tube.geogebra.org/material/iframe/id/1443391/width/800/height/503/border/888888/rc/false/ai/false/sdz/true/smb/false/stb/false/stbh/true/ld/false/sri/true/at/auto" width="800px" height="503px" style="border:0px;"> </iframe>}

Exercise 1

{\bf {Instructions:}} Slide points $A$, $B$ and $C$ until the three sides of the triangle have lengths 5, 7 and 8. 

\begin{question}
How many different triangles can be produced?
\begin{multipleChoice}
\choice{There are no such triangles.}
\choice[correct]{There is one unique triangle.}
\choice{There are two distinct triangles.}
\choice{There are many such triangles.}
\end{multipleChoice}

\begin{question}
(Note: Round angles to the nearest $0.1^\circ$)

What is the degree measure of the largest angle? $\answer[tolerance=0.5]{81.8}^\circ$

What is the degree measure of the smallest angle? $\answer[tolerance=0.5]{38.2}^\circ$

What is the degree measure of the mid-sized angle? $\answer[tolerance=0.5]{59.9}^\circ$
\end{question}
\end{question}

\bigskip

Exercise 2

Slide points $A$, $B$ and $C$ until the three sides of the triangle have lengths 3, 4 and 8. 

\begin{question}
How many different triangles can be produced?
\begin{multipleChoice}
\choice[correct]{There are no such triangles.}
\choice{There is one unique triangle.}
\choice{There are two distinct triangles.}
\choice{There are many such triangles.}
\end{multipleChoice}

\end{question}

\bigskip

Exercise 3

Slide points $A$, $B$ and $C$ until the three angles of the triangle have measures $52^\circ$, $85^\circ$ and $43^\circ$. 

\begin{question}
How many different triangles can be produced?
\begin{multipleChoice}
\choice{There are no such triangles.}
\choice{There is one unique triangle.}
\choice{There are two distinct triangles.}
\choice[correct]{There are many such triangles.}
\end{multipleChoice}

\end{question}

\bigskip

Exercise 4

Slide points $A$, $B$ and $C$ until one angle of the triangle has measure $73^\circ$ and the sides on either side of the $73^\circ$ angle have lengths 4 and 6.

\begin{question}
How many different triangles can be produced?
\begin{multipleChoice}
\choice{There are no such triangles.}
\choice[correct]{There is one unique triangle.}
\choice{There are two distinct triangles.}
\choice{There are many such triangles.}
\end{multipleChoice}
\begin{question}
(Note: Round angles to the nearest $0.1^\circ$. Round lengths to the nearest 0.01)

What is the degree measure of the larger remaining angle? $\answer[tolerance=0.5]{81.8}^\circ$

What is the degree measure of the smaller remaining angle? $\answer[tolerance=0.5]{59.9}^\circ$

What is the length of the remaining side? $\answer[tolerance=0.05]{38.2}^\circ$
\end{question}
\end{question}

Exercise 5

Slide points $A$, $B$ and $C$ until two angles of the triangle have measures $84^\circ$ and $63^\circ$, and the side between the two given angles has length 7.

\begin{question}
How many different triangles can be produced?
\begin{multipleChoice}
\choice{There are no such triangles.}
\choice[correct]{There is one unique triangle.}
\choice{There are two distinct triangles.}
\choice{There are many such triangles.}
\end{multipleChoice}
\begin{question}
(Note: Round angles to the nearest $0.1^\circ$. Round lengths to the nearest 0.01)

What is the degree measure of the remaining angle? $\answer[tolerance=0.5]{33}^\circ$

What is the degree measure of the smaller remaining angle? $\answer[tolerance=0.5]{3}^\circ$

What is the length of the remaining side? $\answer[tolerance=0.05]{3}^\circ$
\end{question}
\end{question}

\end{document}


