% % University of Minnesota Beamer Template (Modified C. Marron)
\documentclass[11pt,xcolor=dvipsnames]{beamer}
\usepackage{graphicx}
\usepackage[german]{babel}
\usepackage[latin1]{inputenc}
\usepackage{times}
\usepackage[T1]{fontenc}
\usepackage{color}
\usepackage{graphicx}              % to include figures
\usepackage{subfig}		   % to place pictures side by side
\usepackage{amsmath}               % great math stuff
\usepackage{amsfonts}              % for blackboard bold, etc
\usepackage{amsthm}                % better theorem environments
\usepackage{amssymb}
\usepackage{tikz}			% for pictures
\usepackage{pgfplots}
\pgfplotsset{compat=newest}% use newest version

\date[mm] 

% % % Fonts: Feel free to comment/uncomment as needed
%\usepackage{newcent}
\usepackage{palatino}
%\usepackage[T1]{fontenc}
%\usepackage[scaled]{helvet}

\mode<presentation>{%
\usetheme{Warsaw}
% Defines the "official'' UMN colors
\xdefinecolor{GopherMaroon}{HTML}{7A0019}
\xdefinecolor{GopherGold}{HTML}{FFCC33}
\xdefinecolor{GopherLightGold}{HTML}{FFDE7A}
\xdefinecolor{GopherDarkMaroon}{HTML}{5B0013}

\usecolortheme[named=GopherMaroon]{structure}
%\usecolortheme[named=GopherGold]{structure}

\usecolortheme{rose} % Try 'lily' if you want to get rid of block backgrounds
\useoutertheme[subsection=true]{smoothbars}
\setbeamertemplate{headline}{} % Uncomment to get standard 'smoothbars' headers back]
\setbeamertemplate{footline}
{%
  \leavevmode%
\hbox{\begin{beamercolorbox}[wd=.5\paperwidth,ht=2.5ex,dp=1.125ex,leftskip=.3cm plus1fill,rightskip=.3cm]{author in head/foot}%
  \usebeamerfont{author in head/foot} University of Minnesota
  \end{beamercolorbox}%

  \begin{beamercolorbox}[wd=.5\paperwidth,ht=2.5ex,dp=1.125ex,leftskip=.3cm,rightskip=.3cm plus1fil]{logo in head/foot}%
    \usebeamerfont{title in head/foot}\inserttitle
  \end{beamercolorbox}}%
  \vskip0pt%
}
\useinnertheme{rounded}
\usefonttheme[onlylarge]{structurebold}
\setbeamertemplate{items}[circle]
\setbeamertemplate{sections/subsections in toc}[circle]
\setbeamertemplate{note page}[plain2]
\setbeamercolor{frametitle}{fg=GopherGold,bg=GopherMaroon}
\setbeamercolor{section in head/foot}{fg=GopherMaroon,bg=GopherGold}
\setbeamercolor{author in head/foot}{fg=GopherMaroon,bg=GopherGold}
\setbeamercolor{logo in head/foot}{fg=GopherGold,bg=GopherMaroon}
\setbeamercolor{block title}{fg=GopherMaroon,bg=GopherGold}
\setbeamertemplate{navigation symbols}{} %Gets rid of the navigation symbols

%Draw the UMN wordmark in the lower right-hand corner
%\logo{\includegraphics[width=0.6in]{graphics/blockm.png}}
}

\title[Sample Problems] {Sample Problems}
\setbeamercolor{title}{fg=GopherGold}
%\institute{\includegraphics[width=1.5in]{graphics/maroonWMlong3}}

\setbeamercovered{invisible}

\begin{document}

\begin{frame}
	\titlepage
\end{frame}
	
\begin{frame}{Inverse Trig Notation}

Rewrite the following problem

\[ \sin \left (\cos^{-1}  \frac{2}{3} \right)\]
\vfill 
If ...

\bigskip

Find ...
\vfill 

\end{frame}

\begin{frame}{Inverse Trig Notation}

Rewrite the following problem

\[ \sin \left (\cos^{-1}  \frac{2}{3} \right)\]
\vfill 
If $\cos \theta = \frac{2}{3}$

\bigskip

Find $\sin \theta$
\vfill 

\end{frame}	
		\begin{frame}{Inverse Trig Notation}

Rewrite the following problem

\[ \sin \left (\cos^{-1}  \frac{2}{3} + \tan^{-1} \frac{3}{2} \right)\]
\vfill 
If ...

\bigskip

Find ...
\vfill 

\end{frame}

\begin{frame}{Inverse Trig Notation}

Rewrite the following problem

\[ \sin \left (\cos^{-1}  \frac{2}{3} + \tan^{-1} \frac{3}{2} \right)\]
\vfill 
If $\cos A = \frac{2}{3}$ and $\tan B = \frac{3}{2}$

\bigskip

Find $\sin(A+B)$

\vfill 

\end{frame}	
	\begin{frame}{Inverse Trig Notation}

Rewrite the following problem

\[ \sin \left (2 \cos^{-1}  \frac{2}{3} \right)\]
\vfill 
If ...

\bigskip

Find ...
\vfill 

\end{frame}

\begin{frame}{Inverse Trig Notation}

Rewrite the following problem

\[ \sin \left (2 \cos^{-1}  \frac{2}{3} \right)\]
\vfill 
If $\cos \theta = \frac{2}{3}$

\bigskip

Find $\sin 2 \theta$
\vfill 

\end{frame}	
	\begin{frame}{Inverse Trig Notation}

Rewrite the following problem

\[ \sin \left (\frac{1}{2} \cos^{-1}  \frac{2}{3} \right)\]
\vfill 
If ...

\bigskip

Find ...
\vfill 

\end{frame}

\begin{frame}{Inverse Trig Notation}

Rewrite the following problem

\[ \sin \left (\frac{1}{2} \cos^{-1}  \frac{2}{3} \right)\]
\vfill 
If $\cos \theta = \frac{2}{3}$

\bigskip

Find $\sin \frac{1}{2} \theta$
\vfill 

\end{frame}	

\begin{frame}

$
\begin{array}{ccc}
0 < \theta < \frac{\pi}{2}&\qquad& \qquad < \frac{\theta}{2} < \; \\
&& \\
\frac{\pi}{2}  < \theta < \pi &\qquad& \qquad < \frac{\theta}{2} < \; \\
&& \\
\pi < \theta < \frac{3\pi}{2}&\qquad& \qquad < \frac{\theta}{2} < \; \\
&& \\
\frac{3\pi}{2}  < \theta < 2\pi &\qquad& \qquad < \frac{\theta}{2} < \; \\
&& \\
2\pi < \theta < \frac{5\pi}{2}&\qquad& \qquad < \frac{\theta}{2} < \; \\
&& \\
\frac{5\pi}{2}  < \theta < 3\pi &\qquad& \qquad < \frac{\theta}{2} < \; \\
&& \\
3\pi < \theta < \frac{7\pi}{2}&\qquad& \qquad < \frac{\theta}{2} < \; \\
&& \\
\frac{7\pi}{2}  < \theta < 4\pi &\qquad& \qquad < \frac{\theta}{2} < \; \\
&& \\
\end{array}
$
\end{frame}
	

\end{document}